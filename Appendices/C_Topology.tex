% Appendix C

\chapter{Hardware} % Main appendix title
\label{AppendixC} % For referencing this appendix elsewhere, use \ref{AppendixB}
In classical computing the CPU is the central processing unit. The fundamental components of a CPU are the transistors. According to Moore's law, the number of transistors integrated in a micro-controller is doubled every two years. Equivalently, the size of transistor is getting smaller every two year. However, we are approaching a size where quantum effects are starting to appear. For this reason, a quantum processor, aka, QPU, is required. There are different ways of implementing a qubit physically. We summarise some of them in the next table.
\begin{table}[H]
\centering
\begin{tabular}{ |c||c|c|c|c|  }
 \hline
 \multicolumn{2}{|c|}{\textbf{Qubit implementation technologies}} \\
 \hline
 \textbf{Name} & \textbf{Companies}  \\
 \hline
Superconducting qubits         & IBM, D-Wave, Rigetti, IQM \\
Trapped ions         & IONQ, Quantinuum, Oxford Ionics, EleQtron \\
Photons     & Xanadu, Orca Computing, PsiQuantum, Quandela \\
Anyons      & Under Development \\
 \hline
\end{tabular}
\caption{List of current ways of implementing a qubit and the companies that are using it.}
\label{tab:QubitTechnologies}
\end{table}
Concretely, we describe the implementation of a D-Wave superconducting qubit which is the most common type of qubits and the ones implemented in the hardware we use in the present work. Quoting DiVicenzo\,\cite{Divincenzo2000TheComputation}, the following criteria have to be fulfill for the realisation of a quantum computer:
\begin{displayquote}
\begin{itemize}
    \item \textit{A scalable physical system with well characterized qubits.}
    \item \textit{The ability to initialize the state of the qubits to a simple fiducial state, such as} $\ket{000...}$.
    \item \textit{Long relevant decoherence times, much longer than the gate operation time.}
    \item \textit{A "universal" set of quantum gates.}
    \item \textit{A qubit-specific measurement capability.}
\end{itemize}
\end{displayquote}
In next sections we describe the physical implementation of superconducting qubits, precisely the ones that make up a quantum annealer. 
\section{Quantum Annealers: An Overview} 
Quantum annealers are single-purpose quantum computers that solve Ising/QUBO problems. D-Wave started using superconducting qubits of wavelength $d$ -- from where D-wave take its name -- but in 2011 D-Wave's researchers demonstrated a way of implementing synthetic qubits\,\cite{Johnson2011QuantumSpins}, i.e., qubits that interact between each other and whose behaviour is that of the Ising model. Here we show a list of D-Wave's quantum annealer.
\begin{table}[H]
\begin{tabular}{ |c||c|c|c|c|  }
 \hline
 \multicolumn{5}{|c|}{\textbf{D-Wave quantum annealers}} \\
 \hline
 \textbf{Name} & \textbf{Release} & \textbf{No. of physical qubits} & \textbf{Architecture} & \textbf{Connectivity}\\
 \hline
 D-Wave One         & 2011 & 128      & Chimera & 6\\
 D-Wave Two         & 2013 & 512      & Chimera & 6\\
 D-Wave 2$\chi$     & 2015 & 1152     & Chimera & 6\\
 D-Wave 2000Q       & 2017 & 2048     & Chimera & 6\\
 D-Wave Advantage   & 2020 & 5640     & Pegasus & 15\\
 D-Wave Advantage 2 & $\sim$2024 & 7440 & Zephyr  & 20\\
 \hline
\end{tabular}
\caption{Evolution of D-Wave quantum annealers.}
\label{tab:DwaveAnnealers}
\end{table}
Notice that the number of qubits is increasing every few years, almost duplicating its number. However, the number of qubits is not the only factor to care about but the connectivity of that qubits. The connectivity is given by the architecture, it represents the number of different qubits to which a given qubit is connected. Once you formulate a QUBO problem, D-Wave map it into a Ising problem and it tries to embed the problem into the hardware. If a possible embedding is found, then the problem can be executed in a quantum annealer. This embedding depends on the number of variables our system has and on the entangled qubits required. In order to be able to solve a problem with the same number of binary variables as the number of physical qubits of a quantum annealer, the quantum annealer must have a fully connected qubits. We are yet far awy from this fully connected architecture though the connectivity between qubits is increasing every few years as Table\,\ref{tab:DwaveAnnealers} shows.
\section{Quantum Annealers: Physical Implementation}
In this section we show the key components of a quantum annealer and how the are related to finally build a superconducting qubit. A quantum annealer requires of a quantum spin system whose spins and couplings can be controlled by external fields, a implementation of the annealing process and the measurements on each spins so that a state can be determined.
%%%%%%%%%%%%%%%%%
\subsection{Josephson Unions}
Consider two superconductor of electric densities $\rho_{1}$ and $\rho_{2}$ and superconducting phases $\theta_{1}$ and $\theta_{2}$, so that their wave function can be written as
\begin{equation}
\label{eq: WaveEq}
\psi_{1}=\sqrt{\rho_{1}}e^{i\theta_{1}} \quad \text{and} \quad\psi_{2}=\sqrt{\rho_{2}}e^{i\theta_{2}}
\end{equation}
where the Schrodinger equation is satisfied separately
\begin{equation}
i\hbar\frac{\partial \psi_{1}}{\partial t}=U_{1}\psi_{1} \quad \text{and} \quad i\hbar\frac{\partial \psi_{2}}{\partial t}=U_{2}\psi_{2}
\end{equation}
Suppose we bring them closer and separate them by an insulating barrier. Because of the tunnelling effect the superconductors will be coupled
\begin{equation}
i\hbar\frac{\partial \psi_{1}}{\partial t}=U_{1}\psi_{1} + K\psi_{2} \quad \text{and} \quad i\hbar\frac{\partial \psi_{2}}{\partial t}=U_{2}\psi_{2} + K\psi_{1}
\end{equation}
Now, connect bot superconductor to a battery of voltage $V$. Then, a potential difference will appear between both sides of the Josephson's union, $U_{1} - U_{2} =qV$. To simplify the notation, assume that $U_{1} = qV/2$ and $U_{2} = -qV/2$, so
\begin{equation}
i\hbar\frac{\partial \psi_{1}}{\partial t}=\frac{qV}{2}\psi_{1} + K\psi_{2} \quad \text{and} \quad i\hbar\frac{\partial \psi_{2}}{\partial t}=\frac{-qV}{2}\psi_{2} + K\psi_{1}
\end{equation}
Substituting the wave equations Eq.\,\eqref{eq: WaveEq}, yields to
\begin{align}
i\hbar \frac{e^{i\theta_{1}}}{2\sqrt{\rho_{1}}}\frac{\partial \rho_{1}}{\partial t} - \hbar\sqrt{\rho_{1}}e^{i\theta_{1}}\frac{\partial\theta_{1}}{\partial t} = \frac{qV}{2}\sqrt{\rho_{1}}e^{i\theta_{1}} + K\sqrt{\rho_{2}}e^{i\theta_{2}} \\
i\hbar \frac{e^{i\theta_{2}}}{2\sqrt{\rho_{2}}}\frac{\partial \rho_{2}}{\partial t} - \hbar\sqrt{\rho_{2}}e^{i\theta_{2}}\frac{\partial\theta_{2}}{\partial t} = -\frac{qV}{2}\sqrt{\rho_{2}}e^{i\theta_{2}} + K\sqrt{\rho_{1}}e^{i\theta_{1}}
\end{align}
Simplifying, 
\begin{align}
i\hbar \frac{1}{2\sqrt{\rho_{1}}}\frac{\partial \rho_{1}}{\partial t} - \hbar\sqrt{\rho_{1}}\frac{\partial\theta_{1}}{\partial t} = \frac{qV}{2}\sqrt{\rho_{1}} + K\sqrt{\rho_{2}}\left(\cos{\phi} + i\sin{\phi}\right) \\
i\hbar \frac{1}{2\sqrt{\rho_{2}}}\frac{\partial \rho_{2}}{\partial t} - \hbar\sqrt{\rho_{2}}\frac{\partial\theta_{2}}{\partial t} = -\frac{qV}{2}\sqrt{\rho_{2}} + K\sqrt{\rho_{1}}\left(\cos{\phi} - i\sin{\phi}\right)
\end{align}
with $\phi = \theta_{2} - \theta_{1}$, and splitting the real and imaginary part,
\begin{align}
\frac{\partial \rho_{1}}{\partial t} = \frac{2}{\hbar}K\sqrt{\rho_{1}\rho_{2}}\sin{\phi} \\
\frac{\partial \rho_{1}}{\partial t} = -\frac{2}{\hbar}K\sqrt{\rho_{1}\rho_{2}}\sin{\phi} \\
\label{eq: theta1}
\frac{\partial \theta_{1}}{\partial t} = - K\sqrt{\frac{\rho_{2}}{\rho_{1}}}\cos{\phi}-\frac{qV}{\hbar} \\
\label{eq: theta2}
\frac{\partial \theta_{2}}{\partial t} = - K\sqrt{\frac{\rho_{2}}{\rho_{1}}}\cos{\phi}+\frac{qV}{\hbar}
\end{align}
Notice that a current between superconductor will exist as long as the superconducting phase difference $\phi \neq 0$.
The current density is given by
\begin{equation}
J = \dot{\rho}_{1} = \frac{2K}{\hbar}\sqrt{\rho_{1}\rho_{2}}\sin{\phi} \equiv J_{c}\sin{\phi}
\end{equation}
knowing that $I_{c} = J_{c}S$ -- where $S$ is the section of the superconductors -- and assuming $\rho_{1} = \rho_{2} =\rho_{0}$ -- the electronic density of the superconductors -- yields to
\begin{equation}
\tag{First Joshepshon equation}
I =\frac{2K\rho_{0}S}{\hbar}\sin{\phi} \equiv I_{c}\sin{\phi}
\end{equation}
where $I_{c}$ is the critical current.
The key feature of previous expression is that the critical current depends on geometric factors $S$ -- section of semiconductors -- and $K$ -- width of the barrier -- that we can control in a lab.\\\\
Subtracting Eq.\,\eqref{eq: theta1} and Eq.\,\eqref{eq: theta2} yields to
\begin{equation}
\frac{qV}{\hbar} = \frac{\partial\phi}{\partial t}
\end{equation}
Noticing that $q=2e$ -- charge of a Cooper pair -- and $\Phi_{0} = \frac{h}{2e}$ -- superconducting magnetic flux quantum --  allow us to rewrite last expression
\begin{equation}
\tag{Second Joshepshon equation}
V = \frac{\Phi_{0}}{2\pi}\frac{\partial \phi}{\partial t}
\end{equation}
We can compute the energy stored in a Josephson's union, $U$, by integrating the power, $I\cdot V$,
\begin{equation}
U = \int_{0}^{t}IVds = \int_{0}^{t}\frac{I_{c}\Phi_{0}}{2\pi}\sin{\phi}\frac{\partial \phi}{\partial s} = \frac{I_{c}\Phi_{0}}{2\pi}\left(1-\cos{\phi}\right)
\end{equation}
which yields to
\begin{equation}
\tag{Third Joshepshon equation}
U = E_{J}\left(1-\cos{\phi}\right), \quad E_{J} = \frac{I_{c}\Phi_{0}}{2\pi}
\end{equation}
where $E_{J}$ is the Josephson energy.
Finally, a Josephson's union has a capacity $C$ given by
\begin{equation}
\tag{Forth Joshepshon equation}
Q = CV = C \frac{\Phi_{0}}{2\pi}\frac{\partial \phi}{\partial t}
\end{equation}
We can write the energy stored in a Josephson's union as
\begin{equation}
E_{C} = \frac{Q^{2}}{2C} = C\cdot \frac{\Phi_{0}^{2}}{2\left(2\pi\right)^{2}}\left(\frac{\partial \phi}{\partial t}\right)^{2}
\end{equation}
According to Lagrange formalism, we can interpret $\phi$ as a generalised coordinate, then its conjugate generalised momentum is given by
\begin{equation}
\frac{\partial E_{C}}{\partial \left(\partial_{t}\phi\right)} = C \frac{\Phi_{0}^{2}}{\left(2\pi\right)^{2}}\frac{\partial \phi}{\partial t} = \frac{\Phi_{0}}{2\pi}Q = \hbar n
\end{equation}
where $n$ indicate the number of Cooper pairs. This implies that $\hbar n$ and $\phi$ are conjugated variables, so that in the quantisation process
\begin{equation}
\left[\hbar \hat{n},\hat{\phi}\right] = i\hbar
\end{equation}
%%%%%%%%%%%%%%%%%%%%%%%%%%%%%%%%%%%%%%%%%%
% Cooper-pair Box
%%%%%%%%%%%%%%%%%%%%%%%%%%%%%%%%%%%%%%%%%%
\subsection{Cooper-pair Box}
This is the easiest way of building a qubit. It is useful for understanding the flux qubit.
IMAGEN A ALVARO
A superconductor isle is connected to a superconductor reservoir by a Josephson's union of capacity $C$.
Tunnelling effect allow Cooper pairs to go through the reservoir to the isle, so that the excess of Cooper pairs $n$ is a dynamic variable. The transfer rate of Cooper pairs is controlled by a gate potential $V_{g}$. In fact, the charge induced by the potential is
\begin{equation}
n_{g} =\frac{C_{g}V_{g}}{2e}
\end{equation}
The Hamiltonian of the system is
\begin{equation}
\mathcal{H} = E_{C}\left(n - n_{g}\right)^{2} + E_{J}\left(1 - \cos{\phi}\right)
\end{equation}
where $E_{C}$ is the required energy to add a Cooper pair to the isle.
By (eq conjugada) we can consider the phase as the generator of translation in the Cooper pairs number space, 
\begin{equation}
e^{in_{0}\hat{\phi}}\ket{n} = \ket{n + n_{0}}
\end{equation}
Then,
\begin{equation}
\cos{\hat{\phi}} = \frac{1}{2}\left(e^{i\hat{\phi}} + e^{-i\hat{\phi}} \right) = \frac{1}{2}\sum_{n}\ket{n+1}\bra{n} + \ket{n-1}\bra{n} = \frac{1}{2}\sum_{n}\ket{n+1}\bra{n} + \ket{n}\bra{n+1}
\end{equation}
and
\begin{equation}
E_{C}\left(\hat{n} - n_{g}\right) = E_{C}\sum_{n}\left(n -n_{g}\right)^{2}\ket{n}\bra{n}
\end{equation}
%%%%%%%%%%%%%%%%%%%%%%%%%%%%%%%%%%%%%%%%%%
% Flux Qubit
%%%%%%%%%%%%%%%%%%%%%%%%%%%%%%%%%%%%%%%%%%
\subsection{Flux Qubit}
%%%%%%%%%%%%%%%%%%%%%%%%%%%%%%%%%%%%%%%%%%
% Topology and Embedding
%%%%%%%%%%%%%%%%%%%%%%%%%%%%%%%%%%%%%%%%%%
\section{D-Wave Topology and Embedding}