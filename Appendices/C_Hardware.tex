% Appendix C

\chapter{Hardware} % Main appendix title
\label{AppendixC} % For referencing this appendix elsewhere, use \ref{AppendixB}
In this Appendix we give an overview of the current quantum hardware without going deeply into details of its implementation. For a better understanding of the implementation of a superconducting qubit -- the ones used by D-Wave -- see Ref.\,\cite{AlvaroDiazComputacionAdiabatica}.
\section{Quantum Technologies}
In classical computing, the \textit{central processing unit} (CPU) is the part of a classical computer that processes the data and performs calculations. The key components of a CPU are called transistors. According to Moore's law, the number of transistors integrated in a micro-controller is doubled every two years. Equivalently, the size of transistors is getting smaller every two years. However, we are approaching the size where quantum effects are starting to appear and have to be taken into account. For this reason, a \textit{quantum processing unit} (QPU) is required. There are different ways of implementing a qubit physically. We summarise some of them in the next table.
\begin{table}[H]
\centering
\begin{tabular}{ |c|c|c|c|c|  }
 \hline
 \multicolumn{2}{|c|}{\textbf{Qubit implementation technologies}} \\
 \hline
 \textbf{Name} & \textbf{Companies}  \\
 \hline
Superconducting qubits         & IBM, D-Wave, Rigetti, IQM \\
 \hline
Trapped ions         & IONQ, Quantinuum, Oxford Ionics, EleQtron \\
 \hline
Photons     & Xanadu, Orca Computing, PsiQuantum, Quandela \\
 \hline
Anyons      & Microsoft \\
 \hline
\end{tabular}
\caption{List of current ways of implementing a qubit and the companies that are are developing them..}
\label{tab:QubitTechnologies}
\end{table}
Concretely, we are interested in the implementation of D-Wave's superconducting qubits. However, a detailed explanation would be lengthy and we will only provide a simple overview of the historical evolution of D-Wave's annealers in terms of number of qubits and connectivity. More details can be found in Ref.\,\cite{AlvaroDiazComputacionAdiabatica}. \\\\
Quoting DiVicenzo\,\cite{Divincenzo2000TheComputation}, the following criteria have to be fulfilled for the realisation of a quantum computer:
\begin{displayquote}
\begin{itemize}
    \item \textit{A scalable physical system with well characterized qubits.}
    \item \textit{The ability to initialize the state of the qubits to a simple fiducial state, such as} $\ket{000...}$.
    \item \textit{Long relevant decoherence times, much longer than the gate operation time.}
    \item \textit{A "universal" set of quantum gates.}
    \item \textit{A qubit-specific measurement capability.}
\end{itemize}
\end{displayquote}
\section{Quantum Annealers: An Overview} 
Quantum annealers are single-purpose quantum computers that solve Ising/QUBO problems. D-Wave started using using d-wave superconductors -- from where D-Wave takes its name -- but in 2011 D-Wave's researchers demonstrated a way of implementing synthetic qubits\,\cite{Johnson2011QuantumSpins}, i.e., qubits that interact between each other and whose behaviour is that of the Ising model. Here we show a list of D-Wave's quantum annealers.
\begin{table}[H]
\centering
\begin{tabular}{ |c|c|c|c|c|  }
 \hline
 \multicolumn{5}{|c|}{\textbf{D-Wave quantum annealers}} \\
 \hline
 \textbf{Name} & \textbf{Release} & \textbf{No. of physical qubits} & \textbf{Architecture} & \textbf{Connectivity}\\
 \hline
 D-Wave One         & 2011 & 128      & Chimera & 6\\
  \hline
 D-Wave Two         & 2013 & 512      & Chimera & 6\\
  \hline
 D-Wave 2$\chi$     & 2015 & 1152     & Chimera & 6\\
  \hline
 D-Wave 2000Q       & 2017 & 2048     & Chimera & 6\\
  \hline
 D-Wave Advantage   & 2020 & 5640     & Pegasus & 15\\
  \hline
 D-Wave Advantage 2 & $\sim$2024 & 7440 & Zephyr  & 20\\
 \hline
\end{tabular}
\caption{Evolution of D-Wave quantum annealers.}
\label{tab:DwaveAnnealers}
\end{table}
Notice that the number of qubits is increasing every few years, almost duplicating its number. However, the number of qubits is not the only factor to care about but the connectivity of those qubits. The connectivity is given by the architecture, it represents the number of different qubits to which a given qubit is connected. Once a QUBO problem is formulated, D-Wave maps it into a Ising problem and it tries to embed the problem into the hardware. If a possible embedding is found, then the problem can be executed in a quantum annealer. This embedding depends on the number of variables our system has and on the entangled qubits required. In order to be able to solve a problem with the same number of binary variables as the number of physical qubits of a quantum annealer, the quantum annealer must have fully-connected qubits. We are yet far away from this fully-connected architecture, even though the connectivity between qubits is increasing every few years as Table\,\ref{tab:DwaveAnnealers} shows.