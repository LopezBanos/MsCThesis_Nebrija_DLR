% Chapter 4

\chapter{Transmission Expansion Planning by Quantum Annealing} % Main chapter title

\label{Chapter4} % For referencing the chapter elsewhere, use \ref{Chapter1} 


\section{Statement of the problem}
\subsection{Easy example}
\begin{table}[h]
\label{tab:wind_turbine}
\centering
\begin{tabular}{ c | c | c | c }
  \hline			
  $t$ & 0 & 1 & 2 \\
    \hline		
  pu & 0.4 & 0.1 & 0.2 \\
  \hline  
\end{tabular}
\caption{Wind Turbine p-max-pu.}
\end{table}
\begin{itemize}
    \item Wind cost: $0$ 
    \item Annualised cost per generator per MW: $20$
    \item Max power of each generator: $6[MW]$
\end{itemize}
\begin{table}[h]
\label{tab:gas}
\centering
\begin{tabular}{ c | c | c | c }
  \hline			
  $t$ & 0 & 1 & 2 \\
    \hline		
  pu & 1 & 1 & 1 \\
  \hline  
\end{tabular}
\caption{Gas p-max-pu.}
\end{table}
\begin{itemize}
    \item Gas cost: $100$ 
    \item Annualised cost per generator per MW: $30$
    \item Max power of each generator: $50[MW]$
\end{itemize}
\begin{table}[h]
\label{tab:load}
\centering
\begin{tabular}{ c | c | c | c }
  \hline			
  $t$ & 0 & 1 & 2 \\
    \hline		
  pu & 2 & 5 & 0 \\
  \hline  
\end{tabular}
\caption{Wind Turbine p-max-pu.}
\end{table}
\textbf{Variables}:
\begin{itemize}
    \item $x(t)$: Total number of wind turbines at time $t$.
    \item $y(t)$: Total number of gas generators at time $t$.
    \item $\theta_{x}(t)$: Energy from carrier = wind  at time $t$ divided by total possible energy at that time. This means is a percentage of the maximum allowed at that instant of time.
    \item $\theta_{x}(t)$: Energy from carrier = gas  at time $t$ divided by total possible energy at that time. This means is a percentage of the maximum allowed at that instant of time.
\end{itemize}
We are going to discretise the percentage we can ask from generators so we can only demand ${0\%,25\%,50\%,75\%,100\%}$. We can re-write,
\begin{itemize}
    \item $\theta_{x}(t) = \sum_{j=0}^{3}0.25\cdot q_{j}(t)$
    \item $\theta_{y}(t) = \sum_{j=4}^{7}0.25\cdot q_{j}(t)$
\end{itemize}
For three snapshots this means we have $24$ binary variables due to this discretization.\\\\
Assuming the maximum number of generators we can build is 3 per instance:
\begin{align}
    x(t_{0}) = 2^{0}\cdot x_{0} + 2^{1}\cdot x_{1} \\
    x(t_{1}) = 2^{0}\cdot x_{2} + 2^{1}\cdot x_{3} \\
    x(t_{2}) = 2^{0}\cdot x_{4} + 2^{1}\cdot x_{5} \\
\end{align}
\begin{align}
    y(t_{0}) = 2^{0}\cdot y_{0} + 2^{1}\cdot y_{1} \\
    y(t_{1}) = 2^{0}\cdot y_{2} + 2^{1}\cdot y_{3} \\
    y(t_{2}) = 2^{0}\cdot y_{4} + 2^{1}\cdot y_{5} \\
\end{align}
$12$ more variables...\\\\
Finally the objective function is to minimise the total cost,
\begin{align}
    \min_{x,y} \text{cost} = \min_{x,y}\{20\cdot\left[2^{0}\cdot\left(x_{0} + x_{2} + x_{4}\right) + 2^{1}\cdot\left(x_{1} + x_{3} + x_{5}\right) \right]\\
    + 30\cdot\left[2^{0}\cdot\left(y_{0} + y_{2} + y_{4}\right) + 2^{1}\cdot\left(y_{1} + y_{3} + y_{5}\right)\right]\\
    + 100\cdot 0.25 \cdot \left[\left(g_{0} + g_{1} + g_{2} + g_{3}\right) + \left(g_{4} + g_{5} + g_{6} + g_{7}\right) + \left(g_{8} + g_{9} + g_{10} + g_{11}\right)\right]\\
    \cdot\left(50\cdot \left[2^{0}\cdot\left(y_{0} + y_{2} + y_{4}\right) + 2^{1}\cdot\left(y_{1} + y_{3} + y_{5}\right)\right]\right)\}
\end{align}
subject to load constraint,
\begin{align}
    2 - \left[0.25 \cdot \left[\left(g_{0} + g_{1} + g_{2} + g_{3}\right) + \left(g_{4} + g_{5} + g_{6} + g_{7}\right) + \left(g_{8} + g_{9} + g_{10} + g_{11}\right)\right]\right]\\
    \cdot\left(6\cdot \left[2^{0}\cdot\left(y_{0} + y_{2} + y_{4}\right) + 2^{1}\cdot\left(y_{1} + y_{3} + y_{5}\right)\right]\right)\\
    - \left[0.25 \cdot \left[\left(w_{0} + w_{1} + w_{2} + w_{3}\right) + \left(w_{4} + w_{5} + w_{6} + w_{7}\right) + \left(w_{8} + w_{9} + w_{10} + w_{11}\right)\right]\right]\\
    \cdot\left(50\cdot \left[2^{0}\cdot\left(x_{0} + x_{2} + x_{4}\right) + 2^{1}\cdot\left(x_{1} + x_{3} + x_{5}\right)\right]\right) = 0
\end{align}
\textbf{TOTAL NUMBER OF VARIABLES: }
\section{Particular suitable problem (Germany)}


